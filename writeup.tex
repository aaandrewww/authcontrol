\documentclass[10pt]{article}

%AMS-TeX packages
\usepackage{amssymb,amsmath,amsthm} 
%geometry (sets margin) and other useful packages
\usepackage[margin=1.25in]{geometry}
\usepackage{graphicx,ctable,booktabs}
\usepackage{verbatim}
\usepackage{color}
\usepackage{listings}
\lstset{ %
language=C,                % choose the language of the code
basicstyle=\footnotesize,       % the size of the fonts that are used for the code
numbers=left,                   % where to put the line-numbers
numberstyle=\footnotesize,      % the size of the fonts that are used for the line-numbers
stepnumber=1,                   % the step between two line-numbers. If it is 1 each line will be numbered
numbersep=5pt,                  % how far the line-numbers are from the code
backgroundcolor=\color{white},  % choose the background color. You must add \usepackage{color}
showspaces=false,               % show spaces adding particular underscores
showstringspaces=false,         % underline spaces within strings
showtabs=false,                 % show tabs within strings adding particular underscores
frame=single,           % adds a frame around the code
tabsize=2,          % sets default tabsize to 2 spaces
captionpos=b,           % sets the caption-position to bottom
breaklines=true,        % sets automatic line breaking
breakatwhitespace=false,    % sets if automatic breaks should only happen at whitespace
escapeinside={\%*}{*)}          % if you want to add a comment within your code
}

%
%Fancy-header package to modify header/page numbering 
%
\usepackage{fancyhdr}
\pagestyle{fancy}
%\addtolength{\headwidth}{\marginparsep} %these change header-rule width
%\addtolength{\headwidth}{\marginparwidth}
\lhead{CS 261 Project Proposal}
\chead{} 
\rhead{\thepage} 
\lfoot{\small Andrew Johnson and Scott Moore} 
\cfoot{} 
\rfoot{\footnotesize CS 261 Project Proposal} 
\renewcommand{\headrulewidth}{.3pt} 
\renewcommand{\footrulewidth}{.3pt}
\setlength\voffset{-0.25in}
\setlength\textheight{648pt}

\usepackage[square,numbers]{natbib}

%%%%%%%%%%%%%%%%%%%%%%%%%%%%%%%%%%%%%%%%%%%%%%%

\usepackage{trfrac}
\newcommand{\sign}[2]{\ensuremath{#1\;\textsf{signed}\;#2}}
\newcommand{\imp}[2]{\ensuremath{#1 \rightarrow #2}}
\newcommand{\says}[2]{\ensuremath{#1\;\textsf{says}\;#2}}
\newcommand{\confirms}[2]{\ensuremath{#1\;\textsf{con}\;#2}}
\newcommand{\ctxt}[0]{\ensuremath{\Gamma}}
\newcommand{\nil}[0]{\ensuremath{\cdot}}
\newcommand{\bnfsep}[0]{\ensuremath{\quad\mid\quad}}
\newcommand{\entails}[2]{\ensuremath{#1 \vdash #2}}
\newcommand{\pred}[2]{\ensuremath{\textsc{#1}(#2)}}
\newcommand{\subst}[2]{\ensuremath{[#1/#2]}}
\newcommand{\abs}[1]{\ensuremath{\forall x:\;#1}}
\newcommand{\rtcheck}[0]{\ensuremath{\textsf{Runtime check}}}
\newcommand{\with}[1]{\ensuremath{\;(\text{with } #1)}}

\begin{document}

\title{CS 261: Project Proposal}
\author{Andrew Johnson and Scott Moore}
\date{November 15, 2011}

\maketitle

\thispagestyle{empty}

\begin{section}{Introduction}
Authorization logics are a principled technique for implementing rule-based access control mechanisms.
They have several advantages over traditional OS access control mechanisms such as access control lists.
In particular, they allow principals to express access control decisions for which they do not have complete information.
Authorization logics also provide \emph{proof-carrying authorization}. That is, the capability to access a resource carries with it the set of authorization decisions that granted it.
And other stuff too\ldots Since we chose our project late, we didn't spend much time on this section in favor of trying to describe our design/plan in more detail.

We're excited about this project because it combines 1) PL stuff, 2) crypto, and 3) some cool concepts from OS like secure bindings.

\subsection{Related work}
There has been quite a lot of work on authorization logics, particular in the context of distributed systems.
Most recently, \citet{Nexus} developed the Nexus operating system, which uses authorization logic to combine different trust bases for secure systems into a cohesive authorization system.
We plan to incorporate Nexus's method for handling the effect of dynamic state on authorization.

We plan to base our authorization logic on the \citet{Bauer}'s simple authorization logic.
More sophisticated authorization logics are given by \citet{AURA} and \citet{Garg}.
\end{section}

\begin{section}{Design}

We plan to replace the current authorization mechanism for JOS system calls (\textsf{envid2env}) with a new mechanism based on a simple authorization logic.
Our authorization system will use authorization proofs as capabilities and support dynamic authorization decisions through secure bindings.
Because our authorization library will be implemented primarily in user space, it also provides a mechanism for implementing access control for user level resources.

\subsection{Principals}

The principals in our system will be environments, \emph{$env_1 \ldots env_n$}, and a special \emph{kernel} environment.
We believe these are appropriate principals because our authorization decisions apply to system calls, primarily those for manipulating environments.

\subsection{Authorization logic}\label{sec:logic}

We plan to base our logic on the simple logic for access control given in \cite{Bauer}, but extend it to support dynamic authorization decisions. The syntax and semantics of the logic are given below and followed with a high level description.
\\[1em]
\begin{tabular}{llcl}
\emph{formulas} & F & ::= & P \bnfsep \imp{F_1}{F_2} \bnfsep \sign{A}{F} \bnfsep \says{A}{F} \bnfsep \confirms{A}{F} \\
\emph{contexts} & \ctxt & ::= & \nil \bnfsep \ctxt,F \\
\end{tabular}
{
\center
$\trfrac[\;init]{}{\entails{\ctxt,F}{F}}$ \hfill
$\trfrac[\;tauto]{\entails{\ctxt}{F}}{\entails{\ctxt}{\says{A}{F}}}$ \hfill
$\trfrac[\;weaken]{\entails{\ctxt,F_1}{F_2}}{\entails{\ctxt}{\imp{F_1}{F_2}}}$ \hfill
$\trfrac[\;impl]{\entails{\ctxt}{F_1} \quad \entails{\ctxt,F_2}{F_3}}{\entails{\ctxt,\imp{F_1}{F_2}}{F_3}}$ \\[1em]
$\trfrac[\;sign]{\entails{\ctxt,F_1}{\says{A}{F_2}}}{\entails{\ctxt,\sign{A}{F_1}}{\says{A}{F_2}}}$ \hfill
$\trfrac[\;conf]{\entails{\ctxt,F_1}{\says{A}{F_2}}}{\entails{\ctxt,\confirms{A}{F_1}}{\says{A}{F_2}}}$ \hfill
$\trfrac[\;say]{\entails{\ctxt,F_1}{\says{A}{F_2}}}{\entails{\ctxt,\says{A}{F_1}}{\says{A}{F_2}}}$
}

\medskip

The meta variable $A$ in the formulas \textsf{signed}, \textsf{says} and \textsf{confirms} ranges over principals.
$P$ would denotes uninterpreted first-order predicates (e.g., \pred{Trusted}{Env}, \pred{Parent}{Env}).
The formula \imp{F_1}{F_2} denotes implication. 

\sign{A}{F} indicates that $A$ believes $F$ to be true; at run time, it corresponds to a digitally signed assertion.
\says{A}{F} denotes that $F$ is a statement which $A$ believes given the formulae asserted by A and the beliefs of other principals.

\paragraph{Example} Suppose that principal $A$ wishes to delegate its authorization decisions to $B$. Then $A$ would assert \sign{A}{\imp{(\says{B}{\pred{OK}{Env}})}{\pred{OK}{Env}}}. Now for any principal $C$ authorized by $B$ (\says{B}{\pred{OK}{C}}), we can derive \says{A}{\pred{OK}{C}}.

\medskip

\confirms{A}{F} is a new formula we are adding to the logic and corresponds with the authority abstraction in the Nexus system \cite{Nexus}. 
\confirms{A}{F} is a \emph{secure binding} of a run time authorization decision to a principal. \confirms{A}{F} is validated by invoking a decision procedure in $A$ that takes the authorization formula $F$ as input and returns true if $A$ believes $F$.
Crucially, checking the validity $\confirms{A}{F}$ does not create a reusable proof object.
This allows time-sensitive or revocable authorization decisions to be implemented on top of our simple logic without explicitly incorporating time or revocation as primitives.
\end{section}

\begin{section}{Implementation}
We broke implementation into three stages.  We first embedded our logic into the Coq Proof Assistant.  Second, we ported that code to C as a stand-alone library.  Third, we integrated the library into the JOS operating system.
\subsection{Coq}
In order to ensure the correctness and completeness of our logic and proof checker, we first embedded it into the Coq Proof Assistant.  This enabled us to prove that our proof checker was correct and always terminated, as well as work out some of the implementation details in a strongly typed language.  For example, we first defined variables as strings and carried an environment mapping names to values.  This made the proof checker overly complicated and stripping it out led to a far simpler implementation both in Coq and C.  Writing the proof checker in Coq allowed us to prove that it was correct (in fact, using Coq's dependent types, we actually return the Coq proof that the proof checker is correct from the check function).  We therefore can be assured that we did not skip any corner cases, or over-simplify.
\subsection{C Library}
There is no way to directly compile Coq code into C code, so we performed this translation by hand.  This obviously invalidates our proof of correctness.  We gained the fact that our algorithms and data structures were correct, but have no such guarantee for the C implementation.  We attempted to compensate for this by writing a number of tests (now in \textsf{user/proof test.c}).  The C library defines the logic as in section \ref{sec:logic}.  For ease of use we also implemented functions to recursively print formulas and proofs, as well as constructors for primitive proofs, primitive formulas, principals, and for more complex, commonly used, proofs.  
\newline\newline
The context was implemented as a stack of formulas since the proof checker adds a formula to the context and, once it has been used, removes that formula.  Formulas and Proofs are represented as tagged \textsf{union}s of the primitive types together with an \textsf{enum} indicating the current type.  To simplify memory management all constructors perform a deep copy of all input data structures.
\subsection{JOS Integration}
\subsubsection{Library Functions}
\subsubsection{System Calls}
\subsubsection{Proof Sharing}

\end{section}

\begin{section}{Evaluation}
\subsection{Performance}
\subsection{Usability}
\end{section}

\begin{section}{Example: Delegation}
Suppose an environment wishes to accept as ``OK" whatever another environment accepts as ``OK".  This is a fairly common type of proof called delegation.  Consider the following example.  We have a parent environment, $P$, and two child environments, $C_1$ and $C_2$.
$C_2$ delegates to $P$.  $C_1$ uses this fact along with the fact that $\sign{P}{\pred{OK}{C_1}}$ to prove $\says{C_2}{\pred{OK}{C_1}}$.  This will allow $C_1$ to make system calls affecting $C_2$, such as $\textsf{sys\_page\_map}$.
\newline\newline
$
P_1 = \trfrac[\;sign]{\trfrac[\;signed]{\rtcheck}{\sign{C_2}{\abs{\imp{\says{P}{\pred{OK}{v_{0}}}}{\pred{OK}{v_{0}}}}}} \quad \trfrac[\;tauto]{\trfrac[\;init]{\rtcheck}{\abs{\imp{\says{P}{\pred{OK}{v_{0}}}}{\pred{OK}{v_{0}}}}}}{\says{C_2}{\abs{\imp{\says{P}{\pred{OK}{v_{0}}}}{\pred{OK}{v_{0}}}}}}}{\says{C_2}{\abs{\imp{\says{P}{\pred{OK}{v_{0}}}}{\pred{OK}{v_{0}}}}}} \\\\\\\\
P_2 = \trfrac[\;tauto]{\trfrac[\;impl]{\trfrac[\;sign]{\trfrac[\;signed]{\rtcheck}{\sign{P}{\pred{OK}{C_1}}} \quad \trfrac[\;tauto]{\trfrac[\;init]{\rtcheck}{\pred{OK}{C_1}}}{\says{P}{\pred{OK}{C_1}}}}{\says{P}{\pred{OK}{C_1}}} \quad \trfrac[\;init]{\rtcheck}{\imp{\says{P}{\pred{OK}{C_1}}}{\pred{OK}{C_1}}}}{\pred{OK}{C_1}}}{\says{C_2}{\pred{OK}{C_1}}} \\\\\\\\
\text{DELEGATION} = \trfrac[\;spec]{P_1 \quad C_1 \quad P_2}{\says{C_2}{\pred{OK}{C_1}}} \\\\\\\
$
We have added helper functions to create these proofs.
\begin{lstlisting}
// Constructed in C2
Proof del = delegate_from_signed(C2, P, OK);  

// Constructed in parent
Formula pred = formula_pred(OK, principal_pcpl(C1));
Proof  perm = says_from_signed(P, pred);

// Constructed in C2
Proof  use_del = use_delegation(C2, P, C1, OK, del, perm);
\end{lstlisting}
$P_1$ is a proof that $C_2$ delegates to $P$ for the OK predicate, and is constructed in line 2 above.  $P_2$ begins with $\sign{P}{\pred{OK}{C_1}}$ and is then used together with $P_1$ to prove $\says{C_2}{\pred{OK}{C_1}}$.  There are runtime checks required by ``signed" to verify the signatures, and by ``init" to verify that the given statement is in the context (note that this proof begins with an empty context, but that the proof checker adds to its context at runtime).  
\end{section}

\begin{section}{Future Work}
\end{section}

\begin{section}{Summary}
\end{section}


\bibliographystyle{plainnat}
\bibliography{bib}

\end{document}
