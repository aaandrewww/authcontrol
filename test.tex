\documentclass[10pt]{article}

%AMS-TeX packages
\usepackage{amssymb,amsmath,amsthm,amssymb}
\usepackage{pdflscape}
%geometry (sets marginit) and other useful packages
\usepackage[margin=0.5in]{geometry}
\usepackage{graphicx,ctable,booktabs}
\usepackage{verbatim}
\usepackage{color}
\usepackage{listings}
\lstset{ %
language=C,                % choose the language of the code
basicstyle=\footnotesize,       % the size of the fonts that are used for the code
numbers=left,                   % where to put the line-numbers
numberstyle=\footnotesize,      % the size of the fonts that are used for the line-numbers
stepnumber=1,                   % the step between two line-numbers. If it is 1 each line will be numbered
numbersep=5pt,                  % how far the line-numbers are from the code
backgroundcolor=\color{white},  % choose the background color. You must add \usepackage{color}
showspaces=false,               % show spaces adding particular underscores
showstringspaces=false,         % underline spaces within strings
showtabs=false,                 % show tabs within strings adding particular underscores
frame=single,           % adds a frame around the code
tabsize=2,          % sets default tabsize to 2 spaces
captionpos=b,           % sets the caption-position to bottom
breaklines=true,        % sets automatic line breaking
breakatwhitespace=false,    % sets if automatic breaks should only happen at whitespace
escapeinside={\%*}{*)}          % if you want to add a comment within your code
}

%
%Fancy-header package to modify header/page numberinitg 
%
\usepackage{fancyhdr}
\pagestyle{fancy}
%\addtolength{\headwidth}{\marginitparsep} %these change header-rule width
%\addtolength{\headwidth}{\marginitparwidth}
\lhead{CS 261 Project Proposal}
\chead{} 
\rhead{\thepage} 
\lfoot{\small Andrew Johnson and Scott Moore} 
\cfoot{} 
\rfoot{\footnotesize CS 261 Project Proposal} 
\renewcommand{\headrulewidth}{.3pt} 
\renewcommand{\footrulewidth}{.3pt}
\setlength\voffset{-0.25in}
\setlength\textheight{648pt}

\usepackage[square,numbers]{natbib}

%%%%%%%%%%%%%%%%%%%%%%%%%%%%%%%%%%%%%%%%%%%%%%%

\usepackage{trfrac}
\newcommand{\sign}[2]{\ensuremath{#1\;\textsf{signed}\;#2}}
\newcommand{\imp}[2]{\ensuremath{#1 \rightarrow #2}}
\newcommand{\says}[2]{\ensuremath{#1\;\textsf{says}\;#2}}
\newcommand{\confirms}[2]{\ensuremath{#1\;\textsf{con}\;#2}}
\newcommand{\ctxt}[0]{\ensuremath{\Gamma}}
\newcommand{\nil}[0]{\ensuremath{\cdot}}
\newcommand{\bnfsep}[0]{\ensuremath{\quad\mid\quad}}
\newcommand{\entails}[2]{\ensuremath{#1 \vdash #2}}
\newcommand{\pred}[2]{\ensuremath{\textsc{#1}(#2)}}
\newcommand{\subst}[2]{\ensuremath{[#1/#2]}}
\newcommand{\abs}[1]{\ensuremath{\forall x:\;#1}}
\newcommand{\rtcheck}[0]{\ensuremath{\textsf{Runtime check}}}
\newcommand{\with}[1]{\ensuremath{\;(\text{with } #1)}}

\begin{document}
$ 
\text{Prinitcipal } 1 == A \\ 
\text{Prinitcipal } 2 == B \\ 
\text{Prinitcipal } 3 == C \\ 
\text{Prinitcipal } 4 == D \\ 
\text{Prinitcipal } 5 == E \\ 
\text{Prinitcipal } 6 == F \\ 
\text{Prinitcipal } 7 == G \\ 
\text{Predicate } 99 == OK \\ 
\text{Predicate } 100 == ALRIGHT \\\\
\text{PRINTING FORMULAE} \\\\ 
\pred{99}{1} \\\\ 
\imp{\pred{99}{1}}{\pred{100}{2}} \\\\ 
\sign{2}{\pred{99}{1}} \\\\ 
\says{2}{\pred{99}{1}} \\\\ 
\confirms{2}{\pred{99}{1}} \\\\ 
\abs{\pred{99}{v_{0}}} \\\\ 
\text{PRINTING CONTEXT} \\\\ 
\pred{99}{1}\\ 
\pred{99}{7}\\ 
\pred{100}{6}\\ 
\pred{99}{5}\\ 
\pred{100}{4}\\ 
\pred{99}{3}\\ 
\pred{100}{2}\\ 
\pred{99}{1}\\ 
 \\\\ 
\text{PRINTING PROOFS} \\\\ 
\trfrac[\;signed]{\rtcheck}{\sign{2}{\pred{99}{1}}} \\\\\\\\ 
\trfrac[\;confirms]{\rtcheck}{\confirms{2}{\pred{99}{1}}} \\\\\\\\ 
\trfrac[\;init]{\rtcheck}{\pred{99}{1}} \\\\\\\\ 
\trfrac[\;tauto]{\trfrac[\;init]{\rtcheck}{\pred{99}{1}}}{\says{2}{\pred{99}{1}}} \\\\\\\\ 
\trfrac[\;weaken impl]{\trfrac[\;init]{\rtcheck}{\pred{99}{1}}}{\imp{\pred{99}{1}}{\pred{100}{2}}} \\\\\\\\ 
\trfrac[\;impl]{\trfrac[\;init]{\rtcheck}{\pred{99}{1}} \quad \trfrac[\;init]{\rtcheck}{\imp{\pred{99}{1}}{\pred{100}{2}}}}{\pred{100}{2}} \\\\\\\\ 
\trfrac[\;conf]{\trfrac[\;confirms]{\rtcheck}{\pred{100}{2}} \quad \trfrac[\;init]{\rtcheck}{\says{1}{\pred{99}{1}}}}{\says{1}{\pred{99}{1}}} \\\\\\\\ 
\trfrac[\;sign]{\trfrac[\;signed]{\rtcheck}{\pred{100}{2}} \quad \trfrac[\;init]{\rtcheck}{\says{1}{\pred{99}{1}}}}{\says{1}{\pred{99}{1}}} \\\\\\\\ 
\trfrac[\;says]{\trfrac[\;init]{\rtcheck}{\says{1}{\pred{100}{2}}} \quad \trfrac[\;init]{\rtcheck}{\says{1}{\pred{99}{1}}}}{\says{1}{\pred{99}{1}}} \\\\\\\\ 
\trfrac[\;spec]{\trfrac[\;init]{\rtcheck}{\says{1}{\abs{\pred{99}{1}}}} \quad 3 \quad\trfrac[\;init]{\rtcheck}{\says{1}{\pred{100}{2}}}}{\says{1}{\pred{100}{2}}} \\\
$ 
\pagebreak
\newline\newline
\newline\newline
$C_2$ delegates to $P$.  $C_1$ uses this fact along with the fact that $\sign{P}{\pred{OK}{C_1}}$ to prove $\says{C_2}{\pred{OK}{C_1}}$.\newline\newline
$
P_1 = \trfrac[\;sign]{\trfrac[\;signed]{\rtcheck}{\sign{C_2}{\abs{\imp{\says{P}{\pred{OK}{v_{0}}}}{\pred{OK}{v_{0}}}}}} \quad \trfrac[\;tauto]{\trfrac[\;init]{\rtcheck}{\abs{\imp{\says{P}{\pred{OK}{v_{0}}}}{\pred{OK}{v_{0}}}}}}{\says{C_2}{\abs{\imp{\says{P}{\pred{OK}{v_{0}}}}{\pred{OK}{v_{0}}}}}}}{\says{C_2}{\abs{\imp{\says{P}{\pred{OK}{v_{0}}}}{\pred{OK}{v_{0}}}}}} \\\\\\\\
P_2 = \trfrac[\;tauto]{\trfrac[\;impl]{\trfrac[\;sign]{\trfrac[\;signed]{\rtcheck}{\sign{P}{\pred{OK}{C_1}}} \quad \trfrac[\;tauto]{\trfrac[\;init]{\rtcheck}{\pred{OK}{C_1}}}{\says{P}{\pred{OK}{C_1}}}}{\says{P}{\pred{OK}{C_1}}} \quad \trfrac[\;init]{\rtcheck}{\imp{\says{P}{\pred{OK}{C_1}}}{\pred{OK}{C_1}}}}{\pred{OK}{C_1}}}{\says{C_2}{\pred{OK}{C_1}}} \\\\\\\\
\text{DELEGATION} = \trfrac[\;spec]{P_1 \quad C_1 \quad P_2}{\says{C_2}{\pred{OK}{C_1}}} \\\\\\\
$
We have added helper functions to create these proofs.
\begin{lstlisting}
// Constructed in C2
Proof del = delegate_from_signed(C2, P, OK);  

// Constructed in parent
Formula pred = formula_pred(OK, principal_pcpl(C1));
Proof  perm = says_from_signed(P, pred);

// Constructed in C2
Proof  use_del = use_delegation(C2, P, C1, OK, del, perm);
\end{lstlisting}
$P_1$ is a proof that $C_2$ delegates to $P$ for the OK predicate, and is constructed in line 2 above.  $P_2$ begins with $\sign{P}{\pred{OK}{C_1}}$ and is then used together with $P_1$ to prove $\says{C_2}{\pred{OK}{C_1}}$.  There are runtime checks required by ``signed" to verify the signatures, and by ``init" to verify that the given statement is in the context (note that this proof begins with an empty context, but that the proof checker adds to its context at runtime).  
 \end{document}
