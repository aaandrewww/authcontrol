%%%%%%%%%%%%%%%%%%%%%%%%%%%%%%%%%%%%%%%%%%%%%%%
%%%This is a science homework template. Modify the preamble to suit your needs. 
%The junk text is   there for you to immediately see how the headers/footers look at first 
%typesetting.
\documentclass[10pt]{article}

%AMS-TeX packages
\usepackage{amssymb,amsmath,amsthm} 
%geometry (sets margin) and other useful packages
\usepackage[margin=1.25in]{geometry}
\usepackage{graphicx,ctable,booktabs}
\usepackage{verbatim}


%
%Redefining sections as problems
%
\makeatletter
\newenvironment{newsec}{\@startsection
       {section}
       {1}
       {-.2em}
       {-3.5ex plus -1ex minus -.2ex}
       {2.3ex plus .2ex}
       {\pagebreak[3]%forces pagebreak when space is small; use \eject for better results
       \large\bf\noindent{Section }
       }
       }
       {%\vspace{1ex}\begin{center} \rule{0.3\linewidth}{.3pt}\end{center}}
       \begin{center}\large\bf \ldots\ldots\ldots\end{center}}
\makeatother


%
%Fancy-header package to modify header/page numbering 
%
\usepackage{fancyhdr}
\pagestyle{fancy}
%\addtolength{\headwidth}{\marginparsep} %these change header-rule width
%\addtolength{\headwidth}{\marginparwidth}
\lhead{Section \thesection}
\chead{} 
\rhead{\thepage} 
\lfoot{\small\scshape Andrew Johnson and Scott Moore} 
\cfoot{} 
\rfoot{\footnotesize CS 261 Project Propoasl} 
\renewcommand{\headrulewidth}{.3pt} 
\renewcommand{\footrulewidth}{.3pt}
\setlength\voffset{-0.25in}
\setlength\textheight{648pt}

%%%%%%%%%%%%%%%%%%%%%%%%%%%%%%%%%%%%%%%%%%%%%%%

%
%Contents of problem set
%    
\begin{document}

\title{CS 261: Project Proposal}
\author{Andrew Johnson and Scott Moore}
\date{November 15, 2011}

\maketitle

\thispagestyle{empty}

%Example problems
\begin{comment}

\begin{problem}{Problem Title}
Problem Statement
\newline\newline
\textbf{\emph{Solution}}:
\newline\newline
Solution goes here
\end{problem}

\end{comment}

\begin{section}{Goal}
We will implement a file system in JOS with fine-grained access control for structured data.  
If we think of a file with well a well defined format, we would like to be able to define different access levels for different fields.  
This access would be enforced by the file system.  
As a simple example, consider a file consisting of the names of users of a system followed by their usage and their passwords.  
We would like read permissions as follows: the usage should be available to all processes, the names available just to system administration processes, and the passwords only available to login processes. 
We may also want to define different write permissions.  
A system administration process may be allowed to add users without being able to read or write existing data, a password setting process may be able to write, but not read, the password fields.
\end{section}
\begin{section}{Steps}
\begin{enumerate}
\item Define access control primitives for JOS.  
We will start with \emph{open}, \emph{close}, \emph{create}, \emph{read} and \emph{write}, and add \emph{insert} and \emph{remove}.
\item Define access control policies.  
Ignoring the structured data aspect, we will define the policies that can be expressed using the above primitives.  
Access will be granted on a per-program basis, with programs identifiers being derived from their names and text.
\item Express access control policies as parser properties.  
Each file will have an associated parser in which will be embedded the data access policies.
\item 
\end{enumerate}
\end{section}


\end{document}