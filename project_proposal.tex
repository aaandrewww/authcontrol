\documentclass[10pt]{article}

%AMS-TeX packages
\usepackage{amssymb,amsmath,amsthm} 
%geometry (sets margin) and other useful packages
\usepackage[margin=1.25in]{geometry}
\usepackage{graphicx,ctable,booktabs}
\usepackage{verbatim}

%
%Fancy-header package to modify header/page numbering 
%
\usepackage{fancyhdr}
\pagestyle{fancy}
%\addtolength{\headwidth}{\marginparsep} %these change header-rule width
%\addtolength{\headwidth}{\marginparwidth}
\lhead{CS 261 Project Proposal}
\chead{} 
\rhead{\thepage} 
\lfoot{\small Andrew Johnson and Scott Moore} 
\cfoot{} 
\rfoot{\footnotesize CS 261 Project Proposal} 
\renewcommand{\headrulewidth}{.3pt} 
\renewcommand{\footrulewidth}{.3pt}
\setlength\voffset{-0.25in}
\setlength\textheight{648pt}

%%%%%%%%%%%%%%%%%%%%%%%%%%%%%%%%%%%%%%%%%%%%%%%

\usepackage{trfrac}
\newcommand{\sign}[2]{\ensuremath{#1\;\textsf{signed}\;#2}}
\newcommand{\imp}[2]{\ensuremath{#1 \rightarrow #2}}
\newcommand{\says}[2]{\ensuremath{#1\;\textsf{says}\;#2}}
\newcommand{\confirms}[2]{\ensuremath{#1\;\textsf{confirms}\;#2}}
\newcommand{\ctxt}[0]{\ensuremath{\Gamma}}
\newcommand{\nil}[0]{\ensuremath{\cdot}}
\newcommand{\bnfsep}[0]{\ensuremath{\quad\mid\quad}}
\newcommand{\entails}[2]{\ensuremath{#1 \vdash #2}}
\newcommand{\pred}[2]{\ensuremath{\textsc{#1}(#2)}}

\begin{document}

\title{CS 261: Project Proposal}
\author{Andrew Johnson and Scott Moore}
\date{November 15, 2011}

\maketitle

\thispagestyle{empty}

\begin{section}{Abstract}
One of the primary goals of the 
\end{section}

\begin{section}{Introduction}

Mainstream operating systems expose relatively inflexible blah blah blah.

Authorization logics are logical frameworks for implementing rule-based access control mechanisms.



\end{section}

\begin{section}{Design}

We plan to implement an authorization mechanism for jOS system calls based on a simple authorization logic.
Our authorization system will use authorization proofs as capabilities and support dynamic authorization decisions through secure bindings.

\subsection{High-level architecture}

\subsection{Principals}

The principals in our system will be environments, \emph{$env_1 \ldots env_n$}, and a special \emph{kernel} environment.
We believe these are appropriate principals because our authorization decisions apply to system calls, primarily those for manipulating environments.
Furthermore, we are replacing the current authorization mechanism provided by the \textsf{envid2env($\ldots$)} function.

\subsection{Authorization logic}

We plan to base our logic on the simple logic for access control given in \cite{Bauer}, but extend it to support dynamic authorization decisions. The syntax and semantics of the logic are given below and followed with a high level description.
\\[1em]
\begin{tabular}{llcl}
\emph{formulas} & F & ::= & P \bnfsep \imp{F_1}{F_2} \bnfsep \sign{A}{F} \bnfsep \says{A}{F} \bnfsep \confirms{A}{F} \\
\emph{contexts} & \ctxt & ::= & \nil \bnfsep \ctxt,F \\
\end{tabular}
{
\center
$\trfrac[\;init]{}{\entails{\ctxt,F}{F}}$ \hfill
$\trfrac[\;tauto]{\entails{\ctxt}{F}}{\entails{\ctxt}{\says{A}{F}}}$ \hfill
$\trfrac[\;weaken]{\entails{\ctxt,F_1}{F_2}}{\entails{\ctxt}{\imp{F_1}{F_2}}}$ \hfill
$\trfrac[\;impl]{\entails{\ctxt,\imp{F_1}{F_2}}{F_3}}{\entails{\ctxt}{F_1} \quad \entails{\ctxt,F_2}{F_3}}$ \\[1em]
$\trfrac[\;sign]{\entails{\ctxt,\sign{A}{F_1}}{\says{A}{F_2}}}{\entails{\ctxt,F_1}{\says{A}{F_2}}}$ \hfill
$\trfrac[\;conf]{\entails{\ctxt,\confirms{A}{F_1}}{\says{A}{F_2}}}{\entails{\ctxt,F_1}{\says{A}{F_2}}}$ \hfill
$\trfrac[\;say]{\entails{\ctxt,\says{A}{F_1}}{\says{A}{F_2}}}{\entails{\ctxt,F_1}{\says{A}{F_2}}}$
}

\medskip

The meta variable $A$ in the formulas \textsf{signed}, \textsf{says} and \textsf{confirms} ranges over principals.
In a final version of our system, $P$ would denote basic first-order predicates (e.g., \pred{Trusted}{Env}, \pred{Parent}{Env}).
The formula \imp{F_1}{F_2} denotes implication. 

\sign{A}{F} indicates that $A$ believes $F$ to be true; at run time, it corresponds to a digitally signed assertion.
\says{A}{F} denotes that $F$ is a statement which $A$ believes given the formulae asserted by A and the beliefs of other principals.

\paragraph{Example} Suppose that principal $A$ wishes to delegate its authorization decisions to $B$. Then $A$ would assert \sign{A}{\imp{(\says{B}{\pred{OK}{Env}})}{\pred{OK}{Env}}}. Now for any principal $C$ authorized by $B$ (\says{B}{\pred{OK}{C}}), we can derive \says{A}{\pred{OK}{C}}.

\medskip

\confirms{A}{F} is a new formula we are adding to the logic and corresponds with the authority abstraction in the Nexus system \cite{Nexus}. 
\confirms{A}{F} is a \emph{secure binding} of a run time authorization decision to a principal. \confirms{A}{F} is validated by invoking a decision procedure in $A$ that takes the authorization formula $F$ as input and returns true if $A$ believes $F$.
Crucially, checking the validity $\confirms{A}{F}$ does not create a reusable proof object.
This allows time-sensitive or revocable authorization decisions to be implemented on top of our simple logic without explicitly incorporating time or revocation as primitives.
\end{section}

\begin{section}{Implementation plan}

\subsection{Implementation stages}

To simplify our initial implementation, we will restrict $P$ to a single predicate, \pred{OK}{Env}.

\subsection{Implementation challenges}

Much of this will be easier with dynamic memory allocation (heap).  
For this we need \emph{malloc}.  
We also will be using a cryptographic math library to support signatures. 
As far as risk goes, the proof-checker could be tricky to implement, but we have a general idea of how to do it.
\end{section}

\begin{section}{Evaluation sketch}
We would like our implementation to not be much slower for existing authentication (i.e. parent-child) checks.  
We will compare the performance of common cross-environment operations with and without the authorization logic.  
We will also see how the authorization logic scales, plotting performance vs. indirection (i.e. how much more expensive is it to check "A says (B says C)" compared with "A says C").
We also see this as an alternative to IPC-based page sharing, and will compare the performance to that.
\end{section}

\begin{section}{Weekly Schedule}
\begin{enumerate}
\item Nov 15 - 21 Logic definition, authorization logic data structures and proof checker
\item Nov 22 - 29 Digital signatures, malloc
\item Nov 30 - Dec 6 Secure bindings for environment authorizations + evaluation + extras(?) (defined in Design section)
\item Dec 7 - Dec 10 Write-up (more evaluation)
\end{enumerate}
\end{section}

\end{document}
