%%%%%%%%%%%%%%%%%%%%%%%%%%%%%%%%%%%%%%%%%%%%%%%
%%%This is a science homework template. Modify the preamble to suit your needs. 
%The junk text is   there for you to immediately see how the headers/footers look at first 
%typesetting.
\documentclass[10pt]{article}

%AMS-TeX packages
\usepackage{amssymb,amsmath,amsthm} 
%geometry (sets margin) and other useful packages
\usepackage[margin=1.25in]{geometry}
\usepackage{graphicx,ctable,booktabs}
\usepackage{verbatim}


%
%Redefining sections as problems
%
\makeatletter
\newenvironment{newsec}{\@startsection
       {section}
       {1}
       {-.2em}
       {-3.5ex plus -1ex minus -.2ex}
       {2.3ex plus .2ex}
       {\pagebreak[3]%forces pagebreak when space is small; use \eject for better results
       \large\bf\noindent{Section }
       }
       }
       {%\vspace{1ex}\begin{center} \rule{0.3\linewidth}{.3pt}\end{center}}
       \begin{center}\large\bf \ldots\ldots\ldots\end{center}}
\makeatother


%
%Fancy-header package to modify header/page numbering 
%
\usepackage{fancyhdr}
\pagestyle{fancy}
%\addtolength{\headwidth}{\marginparsep} %these change header-rule width
%\addtolength{\headwidth}{\marginparwidth}
\lhead{Section \thesection}
\chead{} 
\rhead{\thepage} 
\lfoot{\small\scshape Andrew Johnson and Scott Moore} 
\cfoot{} 
\rfoot{\footnotesize CS 261 Project Propoasl} 
\renewcommand{\headrulewidth}{.3pt} 
\renewcommand{\footrulewidth}{.3pt}
\setlength\voffset{-0.25in}
\setlength\textheight{648pt}

%%%%%%%%%%%%%%%%%%%%%%%%%%%%%%%%%%%%%%%%%%%%%%%

%
%Contents of problem set
%    
\begin{document}

\title{CS 261: Project Proposal}
\author{Andrew Johnson and Scott Moore}
\date{November 15, 2011}

\maketitle

\thispagestyle{empty}

%Example problems
\begin{comment}

\begin{problem}{Problem Title}
Problem Statement
\newline\newline
\textbf{\emph{Solution}}:
\newline\newline
Solution goes here
\end{problem}

\end{comment}

\begin{section}{Abstract}

We plan to implement flexible access-control for 

\end{section}

\begin{section}{Introduction}
\end{section}

\begin{section}{Design}
\end{section}

\begin{section}{Implementation challenges}
Much of this will be easier with dynamic memory allocation (heap).  
For this we need \emph{malloc}.  
We also will be using a cryptographic math library to support signatures. 
As far as risk goes, the proof-checker could be tricky to implement, but we have a general idea of how to do it.
\end{section}

\begin{section}{Evaluation sketch}
We would like our implementation to not be much slower for existing authentication (i.e. parent-child) checks.  
We will compare the performance of common cross-environment operations with and without the authorization logic.  
We will also see how the authorization logic scales, plotting performance vs. indirection (i.e. how much more expensive is it to check "A says (B says C)" compared with "A says C").
We also see this as an alternative to IPC-based page sharing, and will compare the performance to that.
\end{section}

\begin{section}{Weekly Schedule}
\begin{enumerate}
\item Digital signatures, malloc, define the authorization logic
\item Authorization logic data structures and proof checker
\item Secure bindings for environment authorizations + extras (defined in Design section)
\end{enumerate}
\end{section}

\end{document}